\section{CONCLUSION}
\label{S:conclusion}

To overcome the difficulties associated with the model identification in Model Predictive Control (MPC), we introduce a novel idea for predictive control using data: Data-driven model Predictive Control (DPC). \textcolor[rgb]{0,0,1}{Data-driven control is based on non-physical (black-box) models, therefore they can not be integrated with most of the classical control approaches.} The goal is to create data-driven models that are suitable for receiding horizon control. \textcolor[rgb]{0,0,1}{To this aim, we present two algorithms, based on trees and random forests, to create control-oriented models for DPC. We then apply DPC to three different case studies to demonstrate its strength.}
\begin{enumerate}
\item \emph{\textbf{Comparison with MPC.}} We compare the performance of our DPC to MPC on a multivariable bilinear building model. We establish that DPC with random forests shows a remarkable similarity to MPC in the optimal control strategies explaining $70\%$ variance. On the other hand, DPC with regression trees suffers from practical limitations due to model overfitting.
\item \emph{\textbf{Application to Demand Response.}} We further apply DPC with random forests to a large scale 6 story EnergyPlus model with 22 zones for which the traditional model-based control is largely unsuitable due to complex dynamics and the cost of model identification. We show that DPC, relying only on the sensor data, can provide significant energy savings while maintaining thermal comfort. Our results demonstrate that even for such complex system, DPC tracks a reference signal with a mean error of $3\%$.
\item \textcolor[rgb]{0,0,1}{\emph{\textbf{Application to optimal heating system scheduling.}} We demonstrate robustness of our method to uncertainties due to real data acquisition and weather forecast inaccuracies by implementing and testing DPC on historical data from an off-grid house located in L'Aquila, Italy. We derive a predictive model on such real data and design the optimal ON/OFF scheduling for the heating system in order to save energy while guaranteeing thermal comfort for the occupants. We compare the total amount of energy saved with respect to the classical bang-bang controller (widely used in houses for temperature control) using an EnergyPlus model built specifically for the house. We show that we can perform an energy saving that ranges from $25.4\%$ (if we guarantee thermal comfort i.e.~strictly respect the desired temperature range in the rooms) to $49.2\%$ (if we allow small violation in the desired temperature range). Finally, we test the robustness of our method to uncertainties in the real data acquisition and weather forecast.}
\end{enumerate}

DPC has applications which go beyond buildings and energy systems, to industrial process control, and controlling large critical infrastructures like water networks, district heating \& cooling. In general, DPC is immensely valuable in situations where first principles based modeling cost is extremely high.

\subsection{Practical Challenges and Future Work}
\label{SS:challenges}
\begin{enumerate}
\item \emph{\textbf{Data Availability:}} The main practical challenge for DPC lies in the availability of data for training, and we require answers to questions like \emph{how much data (functional testing) is required, and how should the sampling be done?} Therefore, the procedure for optimal experiment design, and model improvement with estimation of variance in predictions is one of the main focus of our ongoing work.
\item \emph{\textbf{Stability:}} While the buildings are inherently stable, many other applications require stability guarantees. In our ongoing work, we are working towards proving asymptotic stability to origin with DPC-RT and DPC-En by using concept of switched LTI systems. This will make DPC useful for systems with faster dynamics.
\end{enumerate}