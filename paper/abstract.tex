Decisions on how to best operate large complex plants such as natural gas processing, oil refineries, and energy efficient buildings are becoming ever so complex that model-based predictive control (MPC) algorithms must play an important role. However, a key factor prohibiting the widespread adoption of MPC, is the cost, time, and effort associated with learning first-principles dynamical models of the underlying physical system. An alternative approach is to employ learning algorithms to build black-box models which rely only on real-time data from the sensors. Machine learning is widely used for regression and classification, but thus far data-driven models have not been used for closed-loop control. We present novel Data Predictive Control (DPC) algorithms that use Regression Trees and Random Forests for receding horizon control. 
%DPC also utilizes Gaussian Processes for smart experiment design by recommending how the accuracy of the current black-box models can be improved. 
We demonstrate the strength of our approach with a case study on a bilinear building model identified using real weather data and sensor measurements. 
In a one-to-one comparison, we show that DPC explains 70\% variation in the MPC controller.
%We validate our results by a one-to-one comparison against a benchmark MPC controller. 
We further apply DPC to a large scale multi-story EnergyPlus building model to curtail total power consumption in a Demand Response setting. 
In such cases, when the model-based controllers fail due to modeling cost, complexity and scalability, our results show that DPC curtails the desired power usage with high confidence.